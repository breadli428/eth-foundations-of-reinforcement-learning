\section*{Instructions}
\begin{itemize}
    \item \emph{Where to submit:} Please submit your solution as a PDF on Moodle. File name should follow the format \texttt{Assignment1-Lastname-Firstname.pdf}. 
    
    \item \emph{How to write solutions:} You should type your solution using LaTeX and following the template. Handwritten solutions will not be graded. Keep in mind the following premise: 
    \begin{itemize}
        \item[-] When writing in English, write short, simple sentences.
        \item[-] When writing a proof, write clear, precise statements. 
    \end{itemize}
    You can use previous points of the same problem without proving them. You can use results from the lectures if you reference them properly.
       
    \item \emph{Discussion:} You may discuss only at a high level with classmates. You should not dig around for homework solutions; if you do rely upon external resources, cite them, and write solutions in your own words. We ask you to please follow the ETH Disciplinary Code. 

    
    \item \emph{Grading:} Grading will be based on the completeness and correctness of your solution according to points assigned for each exercise. \texttt{Final grade = min(regular points + bonus points, 100)}.

    We reserve the right to deduct points on sloppy \LaTeX, minor errors in calculations, and unclear writing in general.
    
    \item \emph{Re-grading:} You may request for regrading within one week after the grade is released, with a written justification of why your solution deserves more points.

     
    \item \emph{Encountering problems?} 
    \begin{itemize}
       \item If you think some exercise is unclear or wrong use the Forum \textit{Assignments} in Moodle or reach out to the TA in charge of the exercise. 
        \item If you have trouble submitting your solution to Moodle within six hours before the deadline due to technical problems, you can send your PDF solution to our head TA. 
    \end{itemize}
\end{itemize}